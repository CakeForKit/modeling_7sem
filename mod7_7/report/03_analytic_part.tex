%\ssr{Введение}
%
%Цель -- придумать и реализовать критерий для оценки случайности числовых последовательностей и разработать программу с графическим интерфейсом. Программа должна анализировать одноразрядные, двухразрядные и трехразрядные числа, полученные из таблиц ГОСТ, алгоритмическим способом и ручным вводом.

\ssr{Теоретическая часть}

В информационный центр приходят клиенты через $10 \pm 2$ минуты. Если все 3 имеющиеся оператора заняты, клиенту отказывают в обслуживании. Операторы имеют разную производительность и могут обеспечивать обслуживание среднего запроса пользователя за:

\begin{itemize}[noitemsep]
	\item Оператор 1: $20 \pm 5$ мин
	\item Оператор 2: $40 \pm 10$ мин  
	\item Оператор 3: $40 \pm 20$ мин
\end{itemize}

Клиенты стараются занять свободного оператора с максимальной производительностью. Полученные запросы сдаются в накопитель, откуда и выбираются на обработку. На 1-й компьютер поступают запросы от 1 и 2 оператора, на 2-й компьютер — от 3-го оператора. Время обработки запросов 1-м и 2-м компьютером составляет 15 и 30 минут соответственно.

%\section*{Переменные модели}

Эндогенные переменные:
\begin{itemize}[noitemsep]
	\item Время обработки задания $i$-м оператором
	\item Время решения задания $j$-м компьютером
\end{itemize}

Экзогенные переменные:
\begin{itemize}[noitemsep]
	\item Число обслуживаемых клиентов
	\item Число клиентов, получивших отказ
\end{itemize}

Вероятность получения отказа:
\[
P_{\text{отказа}} = \frac{N_{\text{rejected}}}{N_{\text{total}}}
\]
где $N_{\text{rejected}}$ -- количество клиентов получивших отказ, $N_{\text{total}}$ -- количество обработанных клиентов и получивших отказ в сумме.

\clearpage
На рисунке~\ref{img:img/model} представлена концептуальная модель системы.
\FloatBarrier
\imgw{0.8\textwidth}{img/model}{Концептуальная модель}
\FloatBarrier
%\imgw{0.8\textwidth}{img/model2}{Модель в терминологии СМО}
%\FloatBarrier


%\clearpage
\ssr{Результат работы программы}
На рисунке~\ref{img:img/p1} приведен результат работы программы. 

\FloatBarrier
\imgw{1\textwidth}{img/p1}{Результат работы программы}
\FloatBarrier
На основе полученных данных можно рассчитать вероятность отказа, которая равна 23\%.


\ssr{Заключение}

В результате выполнения работы была успешно разработана и реализована программа для имитационного моделирования работы информационного центра.
%В результате выполнения работы была успешно разработана и реализована программа для имитационного моделирования системы массового обслуживания. Программа поддерживает два метода моделирования (событийный принцип и принцип $\Delta t$) и различные законы распределения случайных величин.
